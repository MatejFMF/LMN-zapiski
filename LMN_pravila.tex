\documentclass[12pt,a4paper]{article}

    % PREAMBULA


\usepackage{preambola_knap}
\usepackage{xcolor}

\title{Simpsonovi zapiski v \LaTeX  \space  datoteki}

\newcommand{\blue}[1]{\textcolor{blue}{#1}}
\newcommand{\orange}[1]{\textcolor{orange}{#1}}
\newcommand{\purple}[1]{\textcolor{purple}{#1}}
\renewcommand\fbox{\fcolorbox{orange}{white}}
\newcommand{\dokaz}{\orange{\text{\ldots <dokaz> \ldots}}}
\newcommand{\izraz}{\text{<} izraz \text{>}}


%Škatlica(ni zapisana kot \newenvironment, ker je notri \boxed)
\newcommand{\skatlica}[1]{
    \begin{equation*}
        \boxed{
        \begin{aligned}
            #1
        \end{aligned}
        }
    \end{equation*}
}

    % DOKUMENT

\pagenumbering{gobble}
\begin{document}





\maketitle

\break

\section*{Uvod}
Te zapiski so mišljeni kot kompaktna oblika Simpsonovih zapiskov za pravila uporabe in vpeljave.
Namenjena so tudi za uporabo pri kolokvijih, kvizih in izpitih. 
Če kdo najde napako ali ima kakšno opombo me lahko kontaktira. 
(To datoteko bom morda probal povezati z GitHubom.)
Oranžni kvadrati za dokaze izgledajo dokaj vredu, če kdo ve kako narediti to bolje naj prosim sporoči.

\break

    % PRAVILA

\pagenumbering{arabic}

\section{Pravila dokazovanja}
    \subsection*{Pravili zamenjave}
    \begin{itemize}
        \item \emph{izraz} smemo zamenjati z njemu enakim izrazom
        \item \emph{izjavo} smemo zamenjti z njej ekvivalentno izjavo
    \end{itemize}
    za vsak veznik in kvantifikator.

    \subsection*{Pravila vpeljave}
    Povedo nam, kako neposredno dokažemo izjavo s tem veznikom ali kvantifikatorjem.

    \subsection*{Pravila uporabe}
    Povedo nam, kako že znano izjavo uporabimo. 

\section{Konjunkcija}
    \subsection*{Pravilo upeljave}
        \begin{itemize}
            \item Če sta izjavi $ \Phi $ in $ \Psi $ na voljo, potem lahko dodamo v dokaz:
            $$  \Phi \wedge \Psi \blue{\text{ ker veljata }  \Phi  \text{ in } \Psi}  $$
        \end{itemize}

    \subsection*{Pravili uporabe}
    \begin{itemize}
        \item Če je izjava $ \Phi \wedge \Psi $ na voljo, potem lahko dodamo v dokaz:
        $$ \Phi \blue{\text{ ker velja } \Phi \wedge \Psi } $$
        \item Če je izjava $ \Phi \wedge \Psi $ na voljo, potem lahko dodamo v dokaz:
        $$ \Psi \blue{\text{ ker velja } \Phi \wedge \Psi } $$
    \end{itemize}

\section{Implikacija}
    \subsection*{Pravilo vpeljave}
    Lahko dodamo v dokaz:
    $$ \blue{\text{Dokažimo } \Phi \Rightarrow \Psi}  $$
    \begin{equation*}
        \boxed{
        \begin{aligned}
            &\blue{\text{Predpostavimo }} \Phi  \\ 
            &\orange{\text{\ldots <dokaz> \ldots}} \\ 
            &\Psi
        \end{aligned}
        }
    \end{equation*}
    $$ \Phi \Rightarrow \Psi $$

    \noindent \purple{Predpostavka $ \Phi $ je na voljo le v oranžni škatlici.}

    \subsection*{Pravilo uporabe}
    \begin{itemize}
        \item Če sta izjavi $ \Phi \Rightarrow \Psi $ in $ \Phi $ na voljo, potem lahko dodamo v dokaz:
        $$  \Psi \blue{\text{ ker veljata } \Phi \Rightarrow \Psi \text{ in } \Phi}   $$
    \end{itemize}

\section{Disjunkcija}
    \subsection*{Pravili vpeljave}
    \begin{itemize}
        \item Če je izjava $\Phi$ na voljo, potem lahko dodamo v dokaz:
        $$ \Phi \vee \Psi \blue{\text{ ker velja } \Phi } $$
        \item Če je izjava $\Psi$ na voljo, potem lahko dodamo v dokaz:
        $$ \Phi \vee \Psi \blue{\text{ ker velja } \Psi } $$
    \end{itemize}

    \subsection*{Pravilo uporabe}
    \begin{itemize}
        \item Če je izjava $ \Phi \vee \Psi $ na voljo in bi želeli dokazati $ \rho $, lahko dodamo v dokaz:
        $$ \blue{\text{Dokažemo } \rho \text{ z uporabo } \Phi \vee \Psi } $$
        \skatlica{
            &\blue{\text{Predpostavimo }} \Phi \\
            &\orange{\text{\ldots <dokaz> \ldots}} \\
            &\rho 
        }
        \skatlica{
            &\blue{\text{Predpostavimo }} \Psi \\
            &\orange{\text{\ldots <dokaz> \ldots}} \\
            &\rho 
        }
        $$ \rho $$
    \end{itemize}
    \noindent \purple{Vsaka predpostavka je na voljo le v svoji oranžni škatlici.}

\section{Negacija}
    \subsection*{Pravilo vpeljave}
    \begin{itemize}
        \item Lahko dodamo v dokaz:
        $$   \blue{\text{Dokažemo } \neg \Phi  }   $$
        \skatlica{
            &\blue{\text{Predpostavimo }} \Phi \\
            &\dokaz \\
            &\bot 
        }
        $$ \neg \Phi $$
    \end{itemize}
    \purple{Predpostavka $ \Phi $ je na voljo le v oranžni škatlici.}

    \subsection{Pravilo uporabe}
    \begin{itemize}
        \item Če sta $ \neg \Phi $ in $ \Phi  $ na voljo, lahko dodamo v dokaz:
        $$  \bot \blue{\text{ ker veljata } \neg \Phi \text{ in } \Phi } $$
    \end{itemize}


\section{Neresnica}
    \subsection*{Pravila vpeljave ni}
    \subsection*{Pravilo uporabe}
    \begin{itemize}
        \item Če je $ \bot $ na voljo, lahko dodamo v dokaz:
        $$ \Phi \blue{\text{ zaradi protislovja}}  $$
    \end{itemize}

\section{Resnica}
    \subsection*{Pravilo vpeljave}
    \begin{itemize}
        \item Vedno lahko dodamo v dokaz:
        $$ \top \blue{\text{ očitno}}  $$
    \end{itemize}
    \subsection*{Pravila uporabe ni}

\section{Dokaz s protislovjem}
    \begin{itemize}
            \item Dodamo v dokaz:
            $$ \blue{\text{Dokažemo } \Phi \text{ s protislovjem}} $$
            \skatlica{
                &\blue{\text{Predpostavimo }} \neg \Phi \\ 
                &\dokaz \\ 
                &\bot 
            }
            $$ \Phi $$
    \end{itemize}
    \purple{Predpostavka $ \Phi $ je na voljo le v oranžni škatlici.}

\section{Pravilo izključene tretje možnosti}
    \begin{itemize}
        \item Vedno lahko dodamo v dokaz:
        $$ \Phi \vee \neg \Phi \blue{\text{ LEM}}  $$
    \end{itemize}
    \purple{LEM pomeni "Law of the Excluded Middle".}

\section{Univerzalni kvantifikator}
    \subsection*{Pravilo vpeljave}
    \begin{itemize}
        \item Dodamo v dokaz
        $$ \blue{\text{Dokažemo } \forall x \in X. \: \Phi (X)  } $$
        \skatlica{
            &\blue{\text{Naj bo }} x \in X \\ 
            &\dokaz \\ 
            &\Phi (X)
        }
        $$ \forall x \in X. \: \Phi (X) $$
    \end{itemize}
    \purple{Izjava $ x \in X $ doda spremenljivko $x$ v kontext. $x$ mora biti sveža spremenljivka. $x$ je na voljo le v oranžni škatlici. }

    \subsection*{Pravilo uporabe}
    \begin{itemize}
        \item  Če je $ \forall x \in X. \: \Phi (X) $ na voljo in če vemo da je $ \izraz \in X $, lahko dodamo v dokaz:
        $$ \Phi ( \izraz ) \blue{\text{ ker velja } \forall x \in X. \: \Phi (X) } $$ 
    \end{itemize}
    \purple{Vse proste spremenljivke v $ \izraz $u morajo biti it trenutnega konteksta.}

\section{Eksistenčni kvantifikator}
    \subsection*{Pravilo vpeljave}
    \begin{itemize}
        \item Če je $ \Phi ( \izraz )$ na voljo in vemo, da $ \izraz \in X $, lahko dodamo v dokaz:
        $$ \exists x \in X . \: \Phi \blue{\text{ ker velja } \Phi ( \izraz ) } $$ 
    \end{itemize}
    \purple{Samodejno drži, da so vse proste spremenljivke $\izraz$a iz konteksta, ker je to posledica pogoja, da je $\Phi ( \izraz)$ na voljo.}

    \subsection*{Pravilo uporabe}
    \begin{itemize}
        \item Če je izjava $ \exists x \in X. \: \Phi (X) $ na voljo in želimo dokazati $ \rho $, lahko dodamo v dokaz:
        $$ \blue{\text{Dokažemo } \rho \text{ z uporabo } \exists x \in X. \: \Phi (X) } $$
        \skatlica{
            &\blue{\text{Naj bo }} x \in X \\
            &\blue{\text{Predpostavimo }} \Phi ( x ) \\ 
            &\dokaz \\ 
            &\rho
        }
        $$ \rho $$
    \end{itemize}
    \purple{
        Izjava $x \in X $ doda x v kontekst, kjer je x sveža spremenljivka. 
        Spremenljivka $x$ in predpostavka $ \Phi (x) $ sta na voljo le v oranžnem kvadratku. 
    }


\end{document}

